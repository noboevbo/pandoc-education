\mainlanguage[de]
\usemodule[tikz]
% Enable hyperlinks
\setupinteraction
  [state=start,
  title={Suchen im Internet},
  author={bug},
  style=,
  color=,
  contrastcolor=]

\setuppapersize[A4]

\setuplanguage[de] 
[leftquote=\rightsubguillemot,rightquote=\leftsubguillemot,
leftquotation=\rightguillemot,rightquotation=\leftguillemot]

\setuplayout[
  topspace=1.5cm,
  height=fit,
  headerdistance=-0.7cm,
  % footerdistance=2cm,
  bottomspace=2cm,
  ]

\usesymbols[fontawesome]
\usesymbols[cc]

\definehspace[date][7 em]

\definecolor[schoolcolor][c=1,m=0.68,y=0,k=0.14]
\setupheader[text][
    before={\startframed[frame=off,bottomframe=on,framecolor=schoolcolor]},
    after={\stopframed},
    style=\tfx
    ]
\setupfooter[text][
    style=\tfx
    ]
\setupheadertexts[Suchen im Internet]
\setupheadertexts[TGG11][{\symbol[fontawesome-regular][calendar]\hspace[date]\color[schoolcolor]{pmhs}}]
\setupfootertexts[Seite \currentpage~von \lastpage]
\setupfootertexts[bug][{\symbol[cc][cc]\symbol[cc][by]\symbol[cc][sa]}]

\setupfloat[figure][default={here, nonumber}]
\setupfloat[table][default={here, split}]

\definemeasure[cslhangindent][1.5em]
\definenarrower[hangingreferences][left=\measure{cslhangindent}]
\definestartstop [cslreferences] [
		before={%
	  \starthangingreferences[left]
      \setupindenting[-\leftskip,yes,first]
      \doindentation
  	},
  	after=\stophangingreferences,
	]

\starttypescript [gillius]
  \definefontsynonym[GilliusADF-Regular]    [file:GilliusADF-Regular]
  \definefontsynonym[GilliusADF-Italic]     [file:GilliusADF-Italic]
  \definefontsynonym[GilliusADF-Bold]       [file:GilliusADF-Bold]
  \definefontsynonym[GilliusADF-BoldItalic] [file:GilliusADF-BoldItalic]
\stoptypescript

\starttypescript [gillius]
  \setups[font:fallback:sans]
  \definefontsynonym[Sans]                [GilliusADF-Regular]    [features=default]
  \definefontsynonym[SansItalic]          [GilliusADF-Italic]     [features=default]
  \definefontsynonym[SansBold]            [GilliusADF-Bold]       [features=default]
  \definefontsynonym[SansBoldItalic]      [GilliusADF-BoldItalic] [features=default]
\stoptypescript

\starttypescript [gillius]
  \definetypeface [gillius]    [rm] [serif] [gillius]    [default]
\stoptypescript
\setupbodyfont[gillius,ss,13pt]
% \setupxtable[frame=off]
% \setupxtable[head][topframe=on,bottomframe=on]
% \setupxtable[body][]
% \setupxtable[foot][bottomframe=on]

\definebar
  [redmarker]
  [color=red,
   rulethickness=1.2em,
   offset=1.2,
   continue=yes,
   order=background]

\definebar
  [yellowmarker]
  [color=yellow,
   rulethickness=1.2em,
   offset=1.2,
   continue=yes,
   order=background]

\definebar
  [greenmarker]
  [color=green,
   rulethickness=1.2em,
   offset=1.2,
   continue=yes,
   order=background]

\definebar
  [bluemarker]
  [color=green,
   rulethickness=1.2em,
   offset=1.2,
   continue=yes,
   order=background]

\usemodule[vim]
\definevimtyping[html][
  syntax=html,
  numbering=yes,
  tab=4,
  escape=command]

% Turn off section numbers by default
\setuphead[chapter, section, subsection, subsubsection, subsubsubsection, subsubsubsubsection][number=no]


\starttext

\section[title={Example information
text},reference={example-information-text}]

This is an info text. Suchmaschinen, wie zum Beispiel {\bf google.de}
oder {\bf duckduckgo.com}, helfen einem oft dabei Fragen zu beantworten
und Probleme schnell und selbstständig zu lösen. Dazu sind aber gezielte
Suchanfragen und eine Bewertung der Suchergebnisse notwendig. Der QR
Code rechts führt zu einem Video, in dem einige Tipps zur Verwendung von
Suchmaschinen gezeigt werden. Sieh dir das Video an und mache dir
Notizen.

Here's a citation (Ludl, Gulde, and Curio 2019).

Or Footnote Style: \footnote{(Ludl, Gulde, and Curio 2019)} Quarks.

Marks, hier gehts weiter.

\section[title={Exercise: Image},reference={exercise-image}]

This is an exercise with an image, which can be referenced as ich
schreib hier See the Abbildung~1. Was ist denn hier los?

\placefigure[][fig:qrcode]{{\bf Abbildung 1:} This is the
caption}{\externalfigure[qrcodeYouTubeVideo.png]}

This is a question: \thinrule

\section[title={Exercise: Code},reference={exercise-code}]

\starttyping
<body>
  <h1>Test</h1>
  <a href="pmhs.de">Link</a>
</body>
\stoptyping

\starttyping
<body>
  /BTEX\yellowmarker{/ETEX<h1>/BTEX}/ETEXTest</h1>
  <a href="pmhs.de">Link</a>
</body>
\stoptyping

\section[title={Exercise: Lines},reference={exercise-lines}]

Put lines for students.

\startitemize[packed]
\item
  Question 1. \thinrule
\item
  Question 2. \fillinrules[n=2]
\item
  Question 3? \fillinrules[n=4]
\stopitemize

\startitemize[packed]
\item
  Answer 1
\item
  Answer 2
\item
  Answer 3
\stopitemize

\section[title={Exercise: Textbox},reference={exercise-textbox}]

Adding a textbox for student answers:

~

\framed[width=broad, height=5cm]{}

\section[title={Exercise: Multiple
choice},reference={exercise-multiple-choice}]

This is a multiple choice question

\startxtable[option=stretch, frame=off]\startxrow\startxcell \symbol[fontawesome-regular][check-square]~Choice A \stopxcell\startxcell \symbol[fontawesome-regular][square]~Choice B \stopxcell\startxcell \symbol[fontawesome-regular][square]~Choice C \stopxcell\stopxrow\stopxtable

\section[title={Exercise: Table},reference={exercise-table}]

TODO

\section[title={Exercise: Querformat
Tabelle},reference={exercise-querformat-tabelle}]

TODO

\section[title={Lösungen},reference={lösungen}]

\$solutions

\subject[title={References},reference={references}]

\startcslreferences

\reference[ref-ludlSimpleEfficientRealtime2019]{}%
Ludl, Dennis, Thomas Gulde, and Cristóbal Curio. 2019. “Simple yet
Efficient Real-Time Pose-Based Action Recognition.” In {\em 22nd IEEE
Int. Conf. On Intelligent Transportation Systems (ITSC)}, 581--88.
\useURL[url1][https://doi.org/10.1109/ITSC.2019.8917128]\from[url1].

\stopcslreferences

\stoptext
